\section{Conclusions}

Overall, our survey and the SUS showed that users prefer the Distributed Affinity Diagram System (DADS) for exchanging and organizing ideas. In some ways, DADS did exactly what we expected it to do by making user interactions with the system easy and efficient. Based on the number of sources, card moves, and comments in the digital system, we can conclude that digital users are more productive. However, we also found some results that we did not expect. The relationship in our data between navigation, understanding, and flexibility deserves more research to show which of these variables is most important for usability. Despite the preference for digital among users, our research shows that the more moves users make, the less comfortable they are. We also found the systems are no different in promoting users' perception of teamwork. Below we present what we have learned from this study, and what our work means for future groupware researchers and system designers.

There are two areas where the digital system met our expectations with no surprises: Presentation, and Comments.  The data support our hypothesis that the digital presentation tool will make points easier to explain and easier to understand. We also showed that a digital system will allow users to easily organize and document exchanges of comments. The usability(?) of the comment system resulted in a greater number of comments generated in the digital environment. The comment system helped users exchange ideas about points before grouping them. The distributed digital interface made presenting information and exchanging ideas easier for users.

Our results show that users have an easier time navigating in the digital system than the analog system. Across both digital and analog conditions, improved navigation was also significantly positively correlated with higher understanding, which supports our claim that an interface that is more easily navigated will make it easier for users' to understand each other's points. Interestingly, we also discovered that navigation and flexibility have the greatest impact on usefulness. Systems that suffer from poor usefulness or flexibility could try improving navigation to improve the other two factors as well. The relationship between these factors is not yet clear. It may be easy to improve navigation by changing interface design, but flexibility and usefulness are more complex to change. 

DADS users make more point-card movements in the creation of the affinity diagram, resulting in higher satisfaction. Users in the digital environment also spent more time creating the affinity diagram. More movements and longer sessions may be related to higher satisfaction with their results. However, despite more satisfaction with the final result, Digital users do not feel as comfortable when they make more card movements. Because our correlation analysis does not prove the direction of cause for this relationship, future research should explore the connection between satisfaction and movements despite discomfort with the system.

Systems that use shared space without providing private interfaces may have usability problems. For this reason, we designed a distributed interface for the exchange of ideas. However, our data show that the way DADS currently implements distributed screens does not promote teamwork. This was a concern for us because collaboration depends on teamwork, so collaborative work environments should be designed to enhance teamwork. We believe that the digital system's individual screens might prevent higher levels of perceived teamwork, so future experiments will test different screen configurations to see if the single-screen collaborative brainstorming systems promote more teamwork. Future research can explore which parts of distributed interaction systems affect teamwork most.

Improving the usability of the collaborative affinity diagram process was our most important goal for DADS. For a group to adopt a new digital tool, it must be highly usable. The SUS Usability subscore and the overall Usability score are both above the industry average of 68. There is room for improvement in the Learnability subscore. Our analysis of the System Usability Scale results for DADS shows that our system has the potential to be more usable as we increase learnability and improve training materials. 

Our research shows that an integrated system with distributed interfaces can be successful in improving the exchange and organization of ideas. Two different instruments - our own usability survey and the SUS - show that users felt the digital environment is more usable than traditional methods. Based on our research, we believe groupware designers should consider synchronous distributed interfaces to improve presentation, understanding, and idea exchange through user comments, especially when increasing teamwork is not central. Our research explored the relationship between navigation, understanding, and flexibility, but future research should try to find the cause of this relationship when studying groupware, based on our correlation analysis. 

Digital systems for enhancing collaboration have already addressed some of the problems of efficiency and usability that are inherent in analog collaboration tools. However, DADS is an integrated system for digital collaboration that builds on previous work to improve a common way that teams share ideas, making communication more efficient during affinity diagram creation. Our research into the usability of this system provides insight into the way distributed workspaces can improve collaborative discussion by making it easier to share ideas, easier to present and support these ideas with accessible material, and increase satisfaction with the outcome of the discussion. The insights in our detailed usability study can provide new and fruitful directions for researchers in computer-mediated interaction to explore.